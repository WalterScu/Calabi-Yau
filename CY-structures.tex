% !TEX root=./main.tex
At the end of last section, we introduced \textbf{n-Calabi-Yau categories} and weak Calabi-Yau structures on them. Kontsenvich discussed the notion of Calabi-Yau category in setting of $A_\infty$ in his well-known paper \ref{} and Victor Ginzburg defined Calabi-Yau algebras in another way for associative algberas, which is not equivalent to Kontsenvich's definition in general but some cases. Major missions for this section is to compare the difference between these two definitions and try to offer a more general treatment of Calabi-Yau structures.
\par
We need to clarify what a weak Calabi-Yau structure really means before detailed disscution. Suppose $\omega: S \simeq \text{id}_{\mathcal{A}}[n]$ be a given weak $n$-Calabi-Yau structure on $\mathcal{A}$. Kontsenvich conjectured that if $\mathcal{A}$ is good enough then we can find a Hochschild class $[\gamma_{\mathcal{A}}] \in \tx{HH}_n(\mathcal{A})$ such that the induced map $[\gamma_{\mathcal{A}} s^{-n}]: \mathcal{A}^! [n] \rightarrow \mathcal{A}$ is an isomorphism in derived category $D(\mathcal{A}^{e})$, which is proved by Ginzburg in \ref{}. Thus, Hochschild homologies contain all information about Calabi-Yau in our definition.
\subsection{absolute Calabi-Yau structures}
	%!TEX root=../main.tex
A weak Calabi-Yau structure is not strong enough to describe the 'Calabi-Yau properties' for a (DG) category. Now we introduce an enrichment suggested in \ref{}.
\begin{mydefn}
	An \textbf{$n$-Calabi-Yau structure} on a DG-category $\mathcal{A}$ is a negative cyclic class $[\tilde{\gamma_{\mathcal{A}}}] \in HC_n^{-}(\mathcal{A})$ which induces a weak $n$-Calabi-Yau structure on $\mathcal{A}$.
\end{mydefn}  
\subsection{relative Calabi-Yau structures}