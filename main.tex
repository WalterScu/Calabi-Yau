\documentclass[12pt,twoside]{article}
\usepackage[utf8]{inputenc}
\usepackage{times}
% Twelve point fonts are much easier to read than the standard ten point.

% All the packages and set-up are in the Template.sty file, loaded by the following command.
\usepackage{Template}
\usepackage[numbers]{natbib}
\usepackage{enumerate}
%load the package of xy-pic
\usepackage[all,cmtip]{xy}
%load the package to set format of title
\usepackage[center]{titlesec}
%load the package of Diagrams
%\usepackage[nohug,heads=vee]{diagrams}
%\diagramstyle[labelstyle=\scriptstyle]
% Fairly self-explanatory.  Used by \maketitle.

\title{Calabi-Yau categories and Calabi-Yau algebras}
\date{\number\year, \semester}
\author{Haitao,Zou}
%Set the format of titles
\titleformat*{\section}{\centering\Large\bfseries}
\titleformat*{\subsection}{\normalsize\bfseries}
% Begin the actual text of our document
\begin{document}
% The \maketitle command prints title page information at the top of the page.
\maketitle
\begin{abstract}
    These are short notes for Calabi-Yau algebras and Calabi-Yau categories.
\end{abstract}
%
%\thispagestyle{empty}
\pagenumbering{roman}
%\section*{Acknowledgments}
%Acknowledegments is written here.
\tableofcontents
\newpage

\pagenumbering{arabic}
\section{Introduction}
	\input{introduction}
\section{Preliminaries}
	% !TEX root=./main.tex
\subsection{Kan extension}
    
Kan extension is an essential notion in category theory, especially in categorical homotopy theory. In this subsection, we will give a short review of Kan extension and some of its applications. We refer reader to \cite{Rieh2014} and \citep{Quil1967} for more details on Kan extensions and model category.
\par
\begin{mydefn}
Suppose $F: \mathcal{C} \rightarrow \mathcal{D}$ be a given functor. Let $K: \mathcal{C} \rightarrow \mathcal{E}$ be another functor. The \textbf{left Kan extension} of $K$ along with $F$ is a functor $\Lan_{F}K$ such that $\eta : K \Rightarrow \Lan_{F}K \circ F$ is universal i.e. if $\gamma: K \Rightarrow G \circ F$ is a natural transformation, then there exists an unique natural transformation $\delta: \Lan_{F}K \Rightarrow G$ such that $\gamma = (\delta F)\circ \eta$.
In diagram viewpoint, a Kan extension means
\begin{align*}
    \xymatrix{
    \mathcal{C} \ar[rr]^{K}\ar[rd]_{F}&\ar@{=>}[d]_{\gamma}  & \mathcal{E}&\\
    & \mathcal{D} \ar[ru]_{G}& &\\
    }&\xymatrix{
    \mathcal{C} \ar[rr]^{K}\ar[rd]_{F}&\ar@{=>}[d]_{\eta}  & \mathcal{E}&\\
    & \mathcal{D} \ar[ru]|{\Lan_{F}K}\ar@/_2pc/[ru]_{G}^{\exists!\Downarrow}& &\\
    }
\end{align*}
Dually, we can also defined \textbf{right Kan extension} $\Ran_{K}F$ of $K$ along with $F$ as following diagrams
\begin{align*}
    \xymatrix{
    \mathcal{C} \ar[rr]^{K}\ar[rd]_{F}&  & \mathcal{E}&\\
    & \mathcal{D}\ar@{=>}[u]^{\phi} \ar[ru]_{G}& &\\
    }&\xymatrix{
    \mathcal{C} \ar[rr]^{K}\ar[rd]_{F}&  & \mathcal{E}&\\
    & \mathcal{D} \ar@{=>}[u]^{\varepsilon} \ar@/_2pc/[ru]|{\Ran_{F}K}\ar[ru]^{G}_{\exists!\Uparrow}& &\\
    }
\end{align*}
\end{mydefn}

Unfortunately, Kan extension is not always exists in general case though the existence of Kan extension of some functor implies fantastic results.
\par
In Quillen's homotopical algebra theory, Kan extension is used to construct total derived functors.
\begin{mydefn}
Suppose given localizations and functor
\[
\xymatrix{
\mathcal{C}_{1} \ar[r]^{F} \ar[d]^{\gamma_{1}}& \mathcal{C}_{2} \ar[d]^{\gamma_{2}}&\\
S_{1}^{-1}\mathcal{C}_{1}&S_{2}^{-1}\mathcal{C}_{2}
}
\]
The \textbf{left total derived functor} of $F$ is right Kan extension of $\gamma_{2} \circ F$ along with $\gamma_{1}$, denoted by $(\mathbf{L}F, \varepsilon)$, where $\varepsilon$ is the natural transformation in definition.
\end{mydefn}
\begin{prop}[\cite{Quil1969}]
Suppose given localizations and adjoint functors
\[
\xymatrix{
\mathcal{C}_{1} \ar@<1ex>[r]^{F} \ar[d]^{\gamma_{1}}& \mathcal{C}_{2} \ar@<1ex>[l]^{G}_{\bot} \ar[d]^{\gamma_{2}}&\\
S_{1}^{-1}\mathcal{C}_{1}&S_{2}^{-1}\mathcal{C}_{2}
}
\]
such that
\begin{enumerate}
    \item $S_{1}$ contains all isomorphisms of $\mathcal{C}_{1}$. If $f,g$ are maps of $\mathcal{C}_{1}$ such that $gf$ is defined, then if any two of the maps $f$, $g$, $gf$ are in $S_{1}$ so is the third.
    \item A map $f$ in $\mathcal{C}_{2}$ is in $S_{2}$ if and only if $Gf \in S_{1}$.
    \item There exists a functor $R:\mathcal{C}_{1} \rightarrow \mathcal{C}_{1}$ and a natural transformation $\xi R \rightarrow \text{id}$ such that for all $X \in \mathbf{Ob}(\mathcal{C}_{1})$ the maps $\xi: RX \rightarrow X$ and $\beta: RX \rightarrow GFRX$ are in $S_{1}$.
\end{enumerate}
\par
Then the left derived functor $\mathbf{L}F$ exists and is quasi-inverse to the functor $\widetilde{G}: S_{2}^{-1}\mathcal{C}_{2} \rightarrow S_{1}^{-1}\mathcal{C}_{1}$ induced by $G$. In particular $\widetilde{G}$ and $\mathbf{L}F$ are equivalences of categories.
\end{prop}

If $M$ and $N$ are two model categories, we can defined derived functor between them since they have natural localizations to their homotopy categories.
\begin{cor}
If $M$ and $N$ are two model categories, then for any pair of Quillen adjunction $(F,G)$, $\mathbf{L}F$ exists.
\[
\xymatrix{
\mathcal{M} \ar@<1ex>[r]^{F} \ar[d]^{\gamma_{1}}& \mathcal{N} \ar@<1ex>[l]^{G}_{\bot} \ar[d]^{\gamma_{2}}&\\
\mathbf{Ho}\mathcal{M} \ar[r]^{\mathbf{L}F}&\mathbf{Ho}\mathcal{N}
}
\]
\end{cor}
\subsection{Enriched categories and enriched Yoneda lemma}
    % !TEX root=../main.tex
Enriched category theory deals with categories whose morphism sets have additional structures. Always, the structure on the morphism sets makes original category taking more information.
\begin{mydefn}
Suppose $\mathcal{E}$ be a category and $\mathcal{S}$ be a monoidal category. $\mathcal{E}$ is called a enriched category over $S$ if $\mathcal{C}(X,Y)$ is a object in $S$ for all $X,Y$ in $\mathcal{C}$ with following two families of morphisms
\[
\xymatrix{
\circ_{(X,Y,Z)}: \mathcal{E}(Y,Z) \otimes \mathcal{E}(X,Y) \ar[r] & \mathcal{E}(X,Z)& & \forall X,Y,Z \in \textbf{Ob}(\mathcal{E})
}
\]
%
\[
\xymatrix{
u_X : 1 \ar[r] &\mathcal{E}(X,X)& & \forall X \in \textbf{Ob}(\mathcal{E})
}
\]
These morphisms satisfy following properties
%associative
\[
\xymatrix{
(\mathcal{E}(C,D) \otimes \mathcal{E}(B,C))\otimes \mathcal{E}(A,B) \ar[rr]^-{\simeq} \ar[d]_{\circ_{(B,C,D)} \otimes id}& &\mathcal{E}(C,D) \otimes (\mathcal{E}(B,C)\otimes \mathcal{E}(A,B)) \ar[d]^{id \otimes \circ_{(A,B,C)}}&\\ 
\mathcal{E}(B,D) \otimes \mathcal{E}(A,B) \ar[rd]^{\circ_{(A,B,D)}}& &\mathcal{E}(C,D) \otimes \mathcal{E}(A,C) \ar[ld]^{\circ_{(A,C,D)}}&\\
 &\mathcal{E}(A,D)& &
}
\]
%unit
\[
\xymatrix{
& &\mathcal{E}(X,Y) \otimes 1 \ar[d]^{\simeq} \ar[rd]^{u_X}& &\\
& \mathcal{E}(Y,Y) \otimes \mathcal{E}(X,Y) \ar[r]^-{\circ_{(X,Y,Y)}} &\mathcal{E}(X,Y)& \mathcal{E}(X,Y) \otimes \mathcal{E}(X,X) \ar[l]_-{\circ_{(X,X,Y)}}&\\
& &1 \otimes \mathcal{E}(X,Y) \ar[lu]^{u_Y} \ar[u]^{\simeq}& &
}
\]
\end{mydefn}
\begin{ex}
$(\infty, k)$-categories can be defined as $\infty$-categories enriched over $(\infty, k-1)$-categories by induction.
\end{ex}
\begin{ex}
Let $k$ be a field. A \textbf{differential graded category} over $k$ is a category whose morphism sets are all non-negative chain complexes. By the mean, a differential graded category is an enriched category over category $\mathbf{Ch}^{+}(k)$, of non-negative chain complexes over field $k$. It is a essential notion in modern non-commutative geometry.
\end{ex}


\subsection{Some homological algebras}
    Derived category, a notion in homological algebra, is closely related to abelian category. In homotopical view point, a derived category of a abelian category is the localization of chain complexes category over this abelian category with quasi-isomorphisms.
\par
Moreover, we need a notion, a generalization of derived category of an abelian category in classical homological algebra.
\begin{mydefn}
An additive category $\mathcal{C}$ is called a triangulated category if it is with following data:
\begin{enumerate}[(i)]
    \item An additive auto-equivalence functor $T : \mathcal{C}\rightarrow \mathcal{C}$, the functor is called \textbf{shift functor}. For a object $A$ in $\mathcal{C}$, its $n$-shift (i.e by n times actions of $T$ on $A$) is denoted by $A[n]$ or $s^{n}A$, where $n$ is a integer. 
    \item A class of \textbf{distinguished triangles}
    \[
    A \rightarrow B \rightarrow C \rightarrow A[1]
    \]
\end{enumerate}
They satisfy following axioms
\begin{enumerate}[TR1]
    \item $A \xrightarrow{\textbf{id}_{A}} A \rightarrow 0 \rightarrow A[1]$ is distinguished
    \item any morphism $f: A \rightarrow B$ can be completed to a distinguished triangle;
    \item a triangle $A \xrightarrow{u} B \xrightarrow{v} C \xrightarrow{w} A[1]$ is distinguished if and only if 
    \[
    B \xrightarrow{v} C \xrightarrow{w} A[1] \xrightarrow{-u[1]} B[1]
    \]
    is a distinguished triangle;
    \item Any commutative diagram
    \[
    \xymatrix{
    A \ar[r]^{u} \ar[d]^{f} &B \ar[d]^{g} &\\
    A' \ar[r]^{u'}& B' &
    }\]
    extends to a morphism of triangles(i.e a commutative diagram whose rows are distinguished triangles)
    \[
    \xymatrix{
    A \ar[r]^{u} \ar[d]^{f} & B \ar[r]^{v} \ar[d]^{g} & C \ar[r]^{w} \ar[d]^{h} & A[1] \ar[d]^{f[1]}&\\
    A' \ar[r]^{u'}& B' \ar[r]^{v'} & C' \ar[r]^{w'} & A'[1]
    }\]
    \item Given three distinguished triangles
    \begin{align*}
        A \xrightarrow{u} B \xrightarrow{j} C' \xrightarrow{k} A[1]& &B \xrightarrow{v} C \xrightarrow{l} A' \xrightarrow{i} B[1]& &A \xrightarrow{v \circ u} C \rightarrow B' \rightarrow A[1]
    \end{align*}
    there exist two morphisms $f: C' \rightarrow B', g: B' \rightarrow A'$ such that$(\textbf{id}_{A},v,f),(u,\textbf{id}_{C},g)$ define morphisms of the triangles and \[
    \xymatrix{C' \ar[r]^{f} &B' \ar[r]^{g} & A' \ar[r]^{j[1] \circ i}& C'[1]}
    \]
    is a distinguished triangle.
\end{enumerate}
\end{mydefn}

\begin{mydefn}
A commutative square
\[
\xymatrix{
X \ar[r]^{\alpha'} \ar[d]_{\alpha''} &Y' \ar[d]^{\beta'}&\\
Y'' \ar[r]^{\beta''} & Z &
}
\]
is called homotopy cartesian square if there exists an distinguished triangle
\[
\xymatrix{
X \ar[r] &Y' \coprod Y'' \ar[r]& Z \ar[r]^{\gamma} & X[1]
}
\]
$\gamma$ is called a differential of homotopy cartesian square.
\end{mydefn}

\begin{prop}
Diagram 
\[
\xymatrix{
X \ar[r]^{\alpha'} \ar[d]_{\alpha''} &Y' \ar[d]^{\beta'}&\\
Y'' \ar[r]^{\beta''} & Z &
}
\]
is pull-back iff 
\[
\xymatrix{
0 \ar[r] & X \ar[r] & Y' \coprod Y'' \ar[r] & Z \ar[r] & 0&
}
\]
is a distinguished triangle.
\end{prop}

\section{Duality and Serre functor}
    % !TEX root=./main.tex
\subsection{Coherent sheaves category and quasi-coherent sheaves category}
	% !TEX root=../main.tex
\begin{lemma}
\label{lem1}
Suppose $A$ be a commutative ring, $M$ be a $A$-module, then following sequence of $A$-module is exact
\[
0 \rightarrow M \xrightarrow{\alpha} \bigoplus_{i=1}^{n} M_{g_i} \xrightarrow{\beta} \bigoplus_{i,j}M_{g_{i}g_{j}}
\]
where
\begin{align*}
\alpha(m)=(\frac{m}{1},\cdots, \frac{m}{1})& & \beta(\frac{m_1}{g_{1}^{e_1}},\cdots, \frac{m_n}{g_{n}^{e_n}})=(\frac{m_i}{g_{i}^{e_i}}-\frac{m_j}{g_{j}^{e_j}})_{i,j}
\end{align*}
\end{lemma}
\begin{proof}
Clearly, $\alpha$ is injective as $A$-module morphism and $\beta(\alpha(m))=(\frac{m}{1}-\frac{m}{1})_{i,j}=0$. So what is left to verify is that kernel of $\beta$ is contained in the image of $\alpha$, exactly, if $(\frac{m_i}{g_{i}^{e_i}}-\frac{m_j}{g_{j}^{e_j}})_{i,j}=0$, then all $\frac{m_i}{g_{i}^{e_i}}$ equal to $\frac{m}{1}$ for some $m \in M$.
\par
If
\[
\frac{m_i}{g_{i}^{e_i}}-\frac{m_j}{g_{j}^{e_j}}=0, \forall i, j =1,\cdots, n
\]
then there exist $n_{ij} \in \mathcal{N}$ such that
\begin{equation}
\label{eq1}
(g_{i}g_{j})^{n_{ij}}(g_{j}^{e_j} m_i - g_{i}^{e_i} m_j) =0
\end{equation}
Take $N$ be the maximal one of $n_{ij}$, then
\begin{equation}
\label{eq2}
(g_{i}g_{j})^{N}(g_{j}^{e_j} m_i - g_{i}^{e_i} m_j) =0, \forall i,j
\end{equation}
which implies that
\begin{equation}
\label{eq3}
g_{i}^{N}g_{j}^{N+e_{j}}m_{i}=g_{j}^{N}g_{i}^{N+e_{i}} m_{j}
\end{equation}
But there exist $a_i$ that $1=\sum_{i=1}^{n} a_{i} g_{i}$. Let $M=\max \{ N + e_{i} \}$, then
\begin{equation}
\label{eq4}
1=(\sum_{i=1}^{n}a_{i}g_{i})^{nM}=\sum_{i=1}^{n} \tilde{a_{i}}g_{i}^{M}=\sum_{i=1}^{n}\tilde{a_{i}}\tilde{g_{i}}^{k_i} g_{i}^{N+e_{i}}
\end{equation}
Let $b_i = \tilde{a_{i}}\tilde{g}_{i}^{k_i}$. By equations \ref{eq3} and \ref{eq4}, we have
\[
\frac{m_i}{g_{i}^{e_i}}=\frac{n_{i}g_{i}^{N}}{g_{i}^{N+e_{i}}}=\sum_{j=1}^{n}\frac{b_{j}g_{j}^{N+e_{j}}m_{i}g_{i}^{N}}{g_{i}^{N+e_{i}}}=\sum_{j=1}^{n} \frac{b_{j}g_{j}^{N}m_{j}}{1}
\]
Let $m=\sum_{j=1}^{n}b_{j}g_{j}^{N}m_{j}$, then
\[
\frac{m_i}{g_{i}^{e_i}}=\frac{m}{1}, \forall i
\]
Hence the given sequence is exact.
\end{proof}
\begin{mydefn}
Suppose $X$ be affine scheme $\spec A$, $M$ be an $A$-module. Then there is a presheaf over module defined openset-wise as
\[
\widetilde{M}(U)=\{s: U \rightarrow \coprod_{p \in U} M_{p} | s(p) \in M_{p}, s \text{ locally be} \frac{m}{f}, m \in M, f \in A \}
\]
More precisely, $\widetilde{M}$ can be constructed on each standard open as
\[
\widetilde{M}(D(f))=M_{f}
\]
Such presheaf $\widetilde{M}$ is called \textbf{associated presheaf} to $M$.
\end{mydefn}
Remark: Presheaf $\widetilde{M}$ is actually a sheaf. We can check sheaf condition on each standard  open set, which need to verify that following sequence exact
\[
0 \rightarrow \widetilde{M}(D(f)) \xrightarrow{\oplus \rho_{f,g_i}} \bigoplus_{i=1}^{n} \widetilde{M}(D(g_i)) \xrightarrow{\oplus_{i,j} \rho_{g_i,g_{i}g_{j}}-\rho_{g_{j},g_{i}g_{j}}} \bigoplus_{i,j}\widetilde{M}(D(g_i) \cap D_(g_j))
\]
This sequence is equal to
\[
0 \rightarrow M_f \xrightarrow{\alpha} \bigoplus_{i=1}^{n} M_{fg_i} \xrightarrow{\beta} \bigoplus_{i,j}M_{fg_{i}g_{j}}
\]
where $\alpha$ and $\beta$ are defined as in lemma \ref{lem1}. The result is clear from lemma \ref{lem1} by changing $M$ into $M_f$.

\begin{mydefn}
Let $X=\spec A$. The associated sheaf to $A$(as $A$-module) $\widetilde{A}$ is called \textbf{structure sheaf} of $X$ and denoted by $\mathcal{O}_{X}$. More generally, for any scheme $X$ that is covered by affine open sets $\spec A_{i}$, sheaf of $X$ assigning each open set $\spec A_{i}$ associated sheaf to $A_i$ is also called structure sheaf of $X$.
\end{mydefn}

\begin{prop}
Let $X=\spec A$, $A$ be a commutative ring
\begin{enumerate}[(i)]
	\item
	\[
	(-)^{\sim}: A-\mathbf{Mod} \rightarrow \mathbf{Sh(X)}
	\]
	is fully-faithful functor from $A$-modules category to sheaves category of affine scheme $X$.
	\item $\widetilde{(M\otimes_{A}N)} \simeq \widetilde{M} \otimes_{\mathcal{O}_{X}} \widetilde{N}$, where the right tensor is tensor of sheaves over modules.
	\item $\widetilde{(\oplus M_{i})} \simeq \oplus \widetilde{M_{i}}$, where the right direct sum if direct sum of sheaves over modules.
\end{enumerate}
\end{prop}

\begin{mydefn}[Hartshorne, Algebraic Geomtry]
\label{cohdef1}
Let $(X, \mathcal{O}_{X})$ be a ringed space. A sheaf of $\mathcal{O}_{X}$-modules $F$ is \textbf{quasi-coherent} if $X$ can covered by open affine subsets $U_{i} = \spec A_{i}$ such that for each $i$ there is an $A_i$-module $M_i$ with $F|_{U_i} \simeq \widetilde{M_i}$. We say that is \textbf{coherent} if furthermore each $M_i$ can be taken to be a finitely generated $A_i$-module.
\end{mydefn}
It is easy to see that definition \ref{cohdef1} is about scheme. More generally, we can also define quasi-coherent sheaf and coherent sheaf on a arbitrary ringed space.
\begin{mydefn}[Stack Project]
\label{cohdef2}
Let $(X, \mathcal{O}_{X})$ be a ringed space. Let $F$ be a sheaf of $\mathcal{O}_X$-modules. We say that $F$ is \textbf{quasi-coherent} sheaf of $\mathcal{O}_X$-modules if for every point $x\in X$ there exists an open neighbourhood $x \in U \subset X$ such that $F|_{U}$ is isomorphic to the cokernel of map
\[
\bigoplus_{j \in J} \mathcal{O}_{U} \rightarrow \bigoplus_{i \in I} \mathcal{O}_{U}
\]
This means that $X$ is covered by open sets $U$ such that $F|_{U}$ has a presentation of form
\[
\bigoplus_{j \in J} \mathcal{O}_U \rightarrow \bigoplus_{i \in I} \mathcal{O}_U \rightarrow F|_U \rightarrow 0
\]
If quasi-coherent sheaf $F$ satisfies following conditions:
\begin{enumerate}[(i)]
    \item $F$ is of finite type
    \item for every open set  $U \subset X$ and every finite collection $s_i \in F(U)$, the kernel of associated map $\bigoplus_{i} \mathcal{O}_X \rightarrow F|_U$ is of finite type
\end{enumerate}
then $F$ is called \textbf{coherent} sheaf.
\end{mydefn}
\begin{ex}
On any scheme $X$, the structure sheaf $\mathcal{O}_X$ is quasi-coherent and coherent.
\end{ex}

\begin{mydefn}
Let $B$ be a graded ring and let $X= \proj B$. For any $n \in \mathbb{Z}$, we define that sheaf $\mathcal{O}_X(n)$ to be $\widetilde{S(n)}$. We call $\mathcal{O}_{X}(1)$ the twisting sheaf of Serre(or very ample sheaf). For any sheaf of $\mathcal{O}_X$-module, $F$, we denote by $F(n)$ the twisted $F \otimes_{\mathcal{O}_X} \mathcal{O}_{X}(n)$.
\end{mydefn}

\subsection{Differential forms sheaf}
	% !TEX root=../main.tex
\begin{mydefn}
Let $\phi: R \rightarrow S$ be a ring morphism, in other words, $S$ is a $R$-algebra defined by $\phi$. Let $M$ be a $S$-module. A $R$-derivation into $M$ is a map $D: S \rightarrow M$ which satisfies
\begin{enumerate}[(i)]
	\item $D(r_1 s_1 + r_2 s_2)=r_1 D(s_1) + r_2 D(s_2)$
	\item $D(s_1 s_2) = s_1 D(s_2) + D(s_1) s_2$
\end{enumerate}
\end{mydefn}
Set of $R$-derivation into $M$ is denoted by $\der_{R}(S,M)$. It is also a $S$-module.
\[
\der_{R}(S,-): \mathbf{S-Mod} \rightarrow \mathbf{S-Mod}
\]
 is a functor and it is corepresentable. $\Omega_{S/R}^1$ is the co-representing object of functor $\der_{R}(S,-)$, i.e for any $S$-module $M$ there exists a isomorphism
 \[
 \alpha_{M}: \mathbf{S-Mod}(\Omega_{S/R}^1,M) \rightarrow \der_{R}(S,M)
 \]
Let $d= \alpha_{S}^{-1}(\text{id}_{S})$. $(\Omega_{S/R}^1, d)$ is called \textbf{relative differential forms} of $S$ over $R$( also called \textbf{Kahler differential}).

\begin{mydefn}[Modules of differentials]
Let $X$ be a topological space. Let $\phi: \sheaf_{1} \rightarrow \sheaf_{2}$ be a morphism of sheaves of rings. Let $F$ be a $\sheaf_{2}$-module. A $\sheaf_1$-derivation( $\phi$-derivation) into $F$ is a map $D: \sheaf_2 \rightarrow F$ which is $\sheaf_1$-linear and satisfies Leihbniz rule
\[
D(ab)=aD(b)+D(a)b
\]
It implies that $D(1)=D(1\cdot 1)=2D(1)$, so $D(1)=0$ and for all $r \in \phi(\sheaf_1)$, $D(r)=0$.
\par
Analogously, $\der_{\sheaf_1}(\sheaf_2,-)$ is a representable functor and $\Omega_{\sheaf_1/\sheaf_2}^1$ is the representing object of $\der_{\sheaf_1}(\sheaf_2,-)$. $\Omega_{\sheaf_1/\sheaf_2}^1$ is called \textbf{relative differential forms sheaf} of $\sheaf_1$ over $\sheaf_2$.
\end{mydefn}
Actually, in scheme case, the relative differential forms sheaf is compatible as following construction.
\begin{prop}
Let $f: X \rightarrow Y$ be a morphism of schemes. Then there exists a unique quasi-coherent sheaf $\Omega_{X/Y}^1$ on $X$ such that for any affine open subset $V$ of $Y$, any affine open subset $U$ of $ f^{-1}(V)$, and any $x \in U$ we have
\[
\Omega_{X/Y}^{1}|_{U}= (\Omega^{1}_{\mathcal{O}_{X}(U)/\mathcal{O}_{Y}(V)})^{\sim}
\]
\end{prop}
$\sheaf_{2}[F]$ is sheaflication of presheaf $U \mapsto \sheaf_{2}(U)[F(U)]$ where this denotes the free $\sheaf_{2}$-module on the set $F(U)$. For $s \in F(U)$



\section{Calabi-Yau structures and examples}
    % !TEX root=./main.tex
At the end of last section, we introduced \textbf{n-Calabi-Yau categories} and weak Calabi-Yau structures on them. Kontsenvich discussed the notion of Calabi-Yau category in setting of $A_\infty$ in his well-known paper \ref{} and Victor Ginzburg defined Calabi-Yau algebras in another way for associative algberas, which is not equivalent to Kontsenvich's definition in general but some cases. Major missions for this section is to compare the difference between these two definitions and try to offer a more general treatment of Calabi-Yau structures.
\par
We need to clarify what a weak Calabi-Yau structure really means before detailed disscution. Suppose $\omega: S \simeq \text{id}_{\mathcal{A}}[n]$ be a given weak $n$-Calabi-Yau structure on $\mathcal{A}$. Kontsenvich conjectured that if $\mathcal{A}$ is good enough then we can find a Hochschild class $[\gamma_{\mathcal{A}}] \in \tx{HH}_n(\mathcal{A})$ such that the induced map $[\gamma_{\mathcal{A}} s^{-n}]: \mathcal{A}^! [n] \rightarrow \mathcal{A}$ is an isomorphism in derived category $D(\mathcal{A}^{e})$, which is proved by Ginzburg in \ref{}. Thus, Hochschild homologies contain all information about Calabi-Yau in our definition.
\subsection{absolute Calabi-Yau structures}
	%!TEX root=../main.tex
A weak Calabi-Yau structure is not strong enough to describe the 'Calabi-Yau properties' for a (DG) category. Now we introduce an enrichment suggested in \ref{}.
\begin{mydefn}
	An \textbf{$n$-Calabi-Yau structure} on a DG-category $\mathcal{A}$ is a negative cyclic class $[\tilde{\gamma_{\mathcal{A}}}] \in HC_n^{-}(\mathcal{A})$ which induces a weak $n$-Calabi-Yau structure on $\mathcal{A}$.
\end{mydefn}  
\subsection{relative Calabi-Yau structures}
%\section{Koszul duality theory with homotopy viewpoint}
%\section{Notes of some sources}
%    
%\subsection{Twisted tensor product and twisting morphism}
%   Suppose $C \rightarrow A$ is a linear map from coassociative coalgebra to associative algebra and $\Delta$ is coproduct of $C$, $\mu$ is product of $A$. Following composites of morphisms 
\[
C \rightarrow C \otimes C \rightarrow C \otimes A
\]
\[
C \otimes A \rightarrow A \otimes A \rightarrow A
\]
both induce a same derivation on $C \otimes A$.
\[
d^{r}_{\alpha}: C \otimes A \rightarrow C \otimes C \otimes A \rightarrow C \otimes C \otimes A \rightarrow C \otimes A
\]
The construction can be extended to differential graded (co)algebras.
\[
d_{\alpha}:= d_{C\otimes A}+d_{\alpha}^{r}
\]
where $d_{C\otimes A}$ is the derivation of tensor differential graded (co)algebra.
\begin{mydefn}
let $(C, \Delta, \varepsilon)$ be a coalgebra and $(A, \mu, u)$ be an algebra. Suppose $f,g: C \rightarrow A$ is a linear maps. The composte $f * g := \mu \circ (f \otimes g) \circ 
\Delta$ is called the convolution of f and g.
\end{mydefn}
Remark: Let $U: \mathbf{Alg_{k}} \rightarrow \mathbf{Vect_{k}}$ be the forgetful functor sending an algebra to its underlying vector space. The convolution can be viewed as a operation defined on  morphjism set  $\mathbf{Vect_{k}}(U^{op}(C), U(A))$
\begin{mydefn}
(Cartan-Maruer equation) $\partial(\alpha)+ \alpha * \alpha = 0$
\end{mydefn}
\begin{mydefn}
(twisting morphism) If $\alpha \in Hom(C,A)$ and $\alpha$ satisfies Cartan-Maruer equation with degree -1, then we say $\alpha$ is a twisting morphism.
\end{mydefn}
\begin{mydefn}
If $\alpha$ is twisting morphism, then $(C \otimes A , d_{\alpha})$ becomes a complex and is denoted by $C\otimes_{\alpha} A$, where $C$ is differential graded coalgebra, A is differential graded algebra. It's called twisted tensor product of $C$ and $A$ with twsiting morphsim $ \alpha$.
\end{mydefn}
\begin{prop}
(properties of derivations)
\begin{itemize}
    \item $d_{\alpha * \beta}^{r} =d_{\alpha}^{r} \circ d_{\beta}^{r}$
    \item $d_{\alpha}^2=d_{\partial(\alpha)+\alpha * \alpha}^{r}$
\end{itemize}
\end{prop}
\begin{proof}
$d^{r}_{\alpha * \beta} = (id \otimes \mu)(id \otimes \alpha * \beta \otimes id)(\circ \otimes id )=(id \otimes \mu)(id \otimes (\mu(\alpha \otimes \beta) \Delta) \otimes id)(\Delta \otimes id)$
\begin{figure}[htbp]
    \centering
    \includegraphics[width= \textwidth]{pic/proof1.png}
    \caption{draft proof}
    \label{fig:draft proof}
\end{figure}

\end{proof}
By this propsition, it is easy to see that $d_{\alpha}$ is a differential iff $\alpha$ is twisting morphism.
%Examples of Koszul algebra
\subsection{Artin-Shelter regular algebra}
    This is a note on AS regular algebras. We will give some definitions of dimension first.
\begin{mydefn}
Suppose $A$ is a finitely generated graded associative $k$-algebra, $\mathrm{GKdim}(A):=\limsup_{n \to \infty}\log_{n}(\dim V^{n})$ is called \textbf{Gelfand-Kirilov dimension} of $A$.
\end{mydefn}
\begin{mydefn}
Suppose $M$ is a $\mathbb{Z}$-graded right $A$-module. The \textbf{projective dimension} of $M$ $\mathrm{Proj.dim}(M)$ is the minimal $n$ of the length of projective resolution such that 
\[
0 \rightarrow P_{n} \rTo^{d_{n-1}} P_{n-1} \rTo^{d_{n-2}} \cdots \rTo^{d_{1}} P_{1} \rTo^{d_{0}} P_{0} \rTo 0
\]
where $P_{i}$ are all projective graded modules.
The supremum of projection dimension of all left(resp.right) $A$-module is called the left(resp.right) global dimension of $A$ and it's denoted by $\mathrm{l.gl.dim}(A)$. Similiarly, it of all right $A$-module is called the right global dimension of $A$ and denoted by $\mathrm{r.gl.dim}(A)$
\end{mydefn}



\newpage
%%%%%%%%%%%%%%%%%%%%%%%%%%%%%%
%%  The bibliography is here%%
\bibliographystyle{plain}
\bibliography{reference}
\end{document}
