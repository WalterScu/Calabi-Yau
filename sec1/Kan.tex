
Kan extension is an essential notion in category theory, especially in categorical homotopy theory. In this subsection, we will give a short review of Kan extension and some of its applications. We refer reader to \cite{Rieh2014} and \citep{Quil1967} for more details on Kan extensions and model category.
\par
\begin{mydefn}
Suppose $F: \mathcal{C} \rightarrow \mathcal{D}$ be a given functor. Let $K: \mathcal{C} \rightarrow \mathcal{E}$ be another functor. The \textbf{left Kan extension} of $K$ along with $F$ is a functor $\Lan_{F}K$ such that $\eta : K \Rightarrow \Lan_{F}K \circ F$ is universal i.e. if $\gamma: K \Rightarrow G \circ F$ is a natural transformation, then there exists an unique natural transformation $\delta: \Lan_{F}K \Rightarrow G$ such that $\gamma = (\delta F)\circ \eta$.
In diagram viewpoint, a Kan extension means
\begin{align*}
    \xymatrix{
    \mathcal{C} \ar[rr]^{K}\ar[rd]_{F}&\ar@{=>}[d]_{\gamma}  & \mathcal{E}&\\
    & \mathcal{D} \ar[ru]_{G}& &\\
    }&\xymatrix{
    \mathcal{C} \ar[rr]^{K}\ar[rd]_{F}&\ar@{=>}[d]_{\eta}  & \mathcal{E}&\\
    & \mathcal{D} \ar[ru]|{\Lan_{F}K}\ar@/_2pc/[ru]_{G}^{\exists!\Downarrow}& &\\
    }
\end{align*}
Dually, we can also defined \textbf{right Kan extension} $\Ran_{K}F$ of $K$ along with $F$ as following diagrams
\begin{align*}
    \xymatrix{
    \mathcal{C} \ar[rr]^{K}\ar[rd]_{F}&  & \mathcal{E}&\\
    & \mathcal{D}\ar@{=>}[u]^{\phi} \ar[ru]_{G}& &\\
    }&\xymatrix{
    \mathcal{C} \ar[rr]^{K}\ar[rd]_{F}&  & \mathcal{E}&\\
    & \mathcal{D} \ar@{=>}[u]^{\varepsilon} \ar@/_2pc/[ru]|{\Ran_{F}K}\ar[ru]^{G}_{\exists!\Uparrow}& &\\
    }
\end{align*}
\end{mydefn}

Unfortunately, Kan extension is not always exists in general case though the existence of Kan extension of some functor implies fantastic results.
\par
In Quillen's homotopical algebra theory, Kan extension is used to construct total derived functors.
\begin{mydefn}
Suppose given localizations and functor
\[
\xymatrix{
\mathcal{C}_{1} \ar[r]^{F} \ar[d]^{\gamma_{1}}& \mathcal{C}_{2} \ar[d]^{\gamma_{2}}&\\
S_{1}^{-1}\mathcal{C}_{1}&S_{2}^{-1}\mathcal{C}_{2}
}
\]
The \textbf{left total derived functor} of $F$ is right Kan extension of $\gamma_{2} \circ F$ along with $\gamma_{1}$, denoted by $(\mathbf{L}F, \varepsilon)$, where $\varepsilon$ is the natural transformation in definition.
\end{mydefn}
\begin{prop}[\cite{Quil1969}]
Suppose given localizations and adjoint functors
\[
\xymatrix{
\mathcal{C}_{1} \ar@<1ex>[r]^{F} \ar[d]^{\gamma_{1}}& \mathcal{C}_{2} \ar@<1ex>[l]^{G}_{\bot} \ar[d]^{\gamma_{2}}&\\
S_{1}^{-1}\mathcal{C}_{1}&S_{2}^{-1}\mathcal{C}_{2}
}
\]
such that
\begin{enumerate}
    \item $S_{1}$ contains all isomorphisms of $\mathcal{C}_{1}$. If $f,g$ are maps of $\mathcal{C}_{1}$ such that $gf$ is defined, then if any two of the maps $f$, $g$, $gf$ are in $S_{1}$ so is the third.
    \item A map $f$ in $\mathcal{C}_{2}$ is in $S_{2}$ if and only if $Gf \in S_{1}$.
    \item There exists a functor $R:\mathcal{C}_{1} \rightarrow \mathcal{C}_{1}$ and a natural transformation $\xi R \rightarrow \text{id}$ such that for all $X \in \mathbf{Ob}(\mathcal{C}_{1})$ the maps $\xi: RX \rightarrow X$ and $\beta: RX \rightarrow GFRX$ are in $S_{1}$.
\end{enumerate}
\par
Then the left derived functor $\mathbf{L}F$ exists and is quasi-inverse to the functor $\widetilde{G}: S_{2}^{-1}\mathcal{C}_{2} \rightarrow S_{1}^{-1}\mathcal{C}_{1}$ induced by $G$. In particular $\widetilde{G}$ and $\mathbf{L}F$ are equivalences of categories.
\end{prop}

If $M$ and $N$ are two model categories, we can defined derived functor between them since they have natural localizations to their homotopy categories.
\begin{cor}
If $M$ and $N$ are two model categories, then for any pair of Quillen adjunction $(F,G)$, $\mathbf{L}F$ exists.
\[
\xymatrix{
\mathcal{M} \ar@<1ex>[r]^{F} \ar[d]^{\gamma_{1}}& \mathcal{N} \ar@<1ex>[l]^{G}_{\bot} \ar[d]^{\gamma_{2}}&\\
\mathbf{Ho}\mathcal{M} \ar[r]^{\mathbf{L}F}&\mathbf{Ho}\mathcal{N}
}
\]
\end{cor}