% !TEX root=../main.tex
Enriched category theory deals with categories whose morphism sets have additional structures. Always, the structure on the morphism sets makes original category taking more information.
\begin{mydefn}
Suppose $\mathcal{E}$ be a category and $\mathcal{S}$ be a monoidal category. $\mathcal{E}$ is called a enriched category over $S$ if $\mathcal{C}(X,Y)$ is a object in $S$ for all $X,Y$ in $\mathcal{C}$ with following two families of morphisms
\[
\xymatrix{
\circ_{(X,Y,Z)}: \mathcal{E}(Y,Z) \otimes \mathcal{E}(X,Y) \ar[r] & \mathcal{E}(X,Z)& & \forall X,Y,Z \in \textbf{Ob}(\mathcal{E})
}
\]
%
\[
\xymatrix{
u_X : 1 \ar[r] &\mathcal{E}(X,X)& & \forall X \in \textbf{Ob}(\mathcal{E})
}
\]
These morphisms satisfy following properties
%associative
\[
\xymatrix{
(\mathcal{E}(C,D) \otimes \mathcal{E}(B,C))\otimes \mathcal{E}(A,B) \ar[rr]^-{\simeq} \ar[d]_{\circ_{(B,C,D)} \otimes id}& &\mathcal{E}(C,D) \otimes (\mathcal{E}(B,C)\otimes \mathcal{E}(A,B)) \ar[d]^{id \otimes \circ_{(A,B,C)}}&\\ 
\mathcal{E}(B,D) \otimes \mathcal{E}(A,B) \ar[rd]^{\circ_{(A,B,D)}}& &\mathcal{E}(C,D) \otimes \mathcal{E}(A,C) \ar[ld]^{\circ_{(A,C,D)}}&\\
 &\mathcal{E}(A,D)& &
}
\]
%unit
\[
\xymatrix{
& &\mathcal{E}(X,Y) \otimes 1 \ar[d]^{\simeq} \ar[rd]^{u_X}& &\\
& \mathcal{E}(Y,Y) \otimes \mathcal{E}(X,Y) \ar[r]^-{\circ_{(X,Y,Y)}} &\mathcal{E}(X,Y)& \mathcal{E}(X,Y) \otimes \mathcal{E}(X,X) \ar[l]_-{\circ_{(X,X,Y)}}&\\
& &1 \otimes \mathcal{E}(X,Y) \ar[lu]^{u_Y} \ar[u]^{\simeq}& &
}
\]
\end{mydefn}
\begin{ex}
$(\infty, k)$-categories can be defined as $\infty$-categories enriched over $(\infty, k-1)$-categories by induction.
\end{ex}
\begin{ex}
Let $k$ be a field. A \textbf{differential graded category} over $k$ is a category whose morphism sets are all non-negative chain complexes. By the mean, a differential graded category is an enriched category over category $\mathbf{Ch}^{+}(k)$, of non-negative chain complexes over field $k$. It is a essential notion in modern non-commutative geometry.
\end{ex}

